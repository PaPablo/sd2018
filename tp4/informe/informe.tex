\input{project_settings}
\input{listings_settings}

\begin{document}

%%%%%%%%%%%%%%%%%%%%%%%%%%%%%%%%%%%%%%%%%%%%
% En 'titlepage.tex' se encuentra la página de título
%%%%%%%%%%%%%%%%%%%%%%%%%%%%%%%%%%%%%%%%%%%%
\input{titlepage}

%%%%%%%%%%%%%%%%%%%%%%%%%%%%%%%%%%%%%%%%%%%%
% INDICE
%%%%%%%%%%%%%%%%%%%%%%%%%%%%%%%%%%%%%%%%%%%%
\clearpage
\tableofcontents
\clearpage 

\lstset{style=MyStyle}

\section{RFS con JADE}

Se implementó un Remote File System utilizando JADE. Implementando como funcionalidad expuesta al usuario las operaciones de read y write.

En esta versión, debido a cómo opera JADE, ya no tiene sentido hablar de Cliente o Servidor. En cambio se hablará de origen y destino.

En resumen, esta versión de la solución funciona de la siguiente manera:

\begin{itemize}
    \item El origen, que posee los archivos a ofrecer, crea el \textbf{Contenedor Principal}.
    \item El destino, se conecta al origen y requiere tal o cual archivo. Ya sea para leer del origen o escribiendo en él.
\end{itemize}

Para poder leer y escribir, la mecánica consiste en moverse al lugar a operar, leer o escribir un buffer, migrar al otro extremo y hacer la operación inversa.

Esta arquitectura dispuesta nos ofrece lo siguiente:

\begin{itemize}
    \item La noción de cliente y servidor se pierde.
    \item En cuanto a la eficiencia, podemos decir que no aprovecha de la mejor manera, ya que está migrando el agente por cada lectura de buffer.
    \item El hecho de tener un comportamiento cíclico para trabajar hace que se desdibuje toda la arquitectura en capas que se poseía en RMI.
\end{itemize}

En cuanto a la movilidad de nuestra solución. Podríamos decir que la solución con JADE tiene un peso algo mayor, ya que para poder funcionar es necesaria toda la instalación de JADE en ambos sitios para poder funcionar. En cambio, con las soluciones de RMI era necesaria la infraestructura del \texttt{rmiregistry}. Ni hablar de la implementación con sockets, que requería solamente el binario del cliente o del servidor.

\subsection{Funcionamiento de la herrameinta}

\subsubsection{Compilación}

Dentro de la carpeta del proyecto, compilar con \texttt{make} 

\subsubsection{Puesta en funcionamiento}

Para la utilización de la herramienta se debe tener en otro equipo corriendo el \textbf{Main Container} y desde el segundo equipo hacer.

\begin{lstlisting}[breaklines=true]
java -cp lib/jade.jar:classes Main <ip-main-container>\\
    <read|write> <src-file> <dst-file>
\end{lstlisting}

En el caso del \texttt{write} el \texttt{src-file} proviene del equipo con el Main Container, y en el caso del \texttt{read} del equipo que se conecta.

\subsection{Comparación de soluciones}

Luego de haber movido un archivo bastante grande: \texttt{debian-live-9.5.0-amd64-xfce.iso (1.9 GB)} utilizando las distintas implementaciones realizadas a lo largo de la materia, con el mismo tamaño de buffer 8KB, se observaron los siguientes tiempos.

\begin{center}
  \begin{tabular}{ | l | c | }
    \hline
    Java Sockets & 6:30+ minutos \\
    \hline
    Java RMI & 25,64s\\
    \hline
    JADE & 30,41s \\
    \hline
  \end{tabular}
\end{center}

De los tiempos observados podemos concluir qué:

\begin{itemize}
    \item La implementación con sockets es la más lenta, aunque sería la más \textit{a mano} que hay.
    \item Por más que en JADE o RMI el andamiaje es más pesado para poder poner en marcha la funcionalidad, resulta más rápido, porque son soluciones \textit{ya diseñadas}.
    \item JADE puede estar tardando un poco más debido a que tiene que mover el agente entre los contenedores. 
\end{itemize}

\section{Recorrer varias máquinas con JADE}

Se implementó un agente JADE que viaja entre los distintos contenedores que componene la \emph{AP} (\emph{Agent Platform}), y recupera la siguiente información de la máquina que creó el contenedor:

\begin{itemize}
    \item Uso del CPU
    \begin{itemize}
        \item Porcentaje de uso de la JVM actual
        \item Porcentaje de uso en total
    \end{itemize}
    \item Dirección IP
    \item \emph{Hostname} 
    \item Hora del sistema (en milisegundos)
\end{itemize}

Para ello se creó una clase (\texttt{SystemInfoCollector.java}) que se encarga de recuperar dicha información y persistirla como estado del objeto.

El agente se muda por todos los contenedores, se queda en cada uno un total de 10 segundos; una vez que terminó con el recorrido imprime la información recolectada, y el tiempo total transcurrido desde que comenzó.

\section{MQTT}

Para la creación tanto de un tablero de control (\emph{dashboard}), como para la emulación de dispositivos que publican información en varios tópicos, se utilizó la herramiento \textbf{Node-RED}\footnote{Editor \emph{web} basado en flujos de datos\autocite{NodeRED}}. Como \emph{broker} MQTT se utilizó el servidor \textbf{mosquitto}.

Para facilitar la ejecución tanto de Node-RED como del \emph{broker}, se confeccionó un archivo de configuración de \emph{docker-compose}, el cual se encarga de levantar un servicio para el servidor de Node-RED, y otro para el \emph{broker}.   

\subsection{Corriendo los servicios}

\begin{itemize}
    \item Levantar los contenedores de \emph{Docker} ejecutando el siguiendo comando (dentro del mismo directorio donde se encuentre el archivo \texttt{docker-compose.yml})
    \begin{itemize}
        \item \texttt{docker-compose --build -d up} 
    \end{itemize}
    \item Dirigirse a la dirección \texttt{localhost:1880} para acceder a la aplicación Node-RED
    \item Abrir el menú ubicado en la esquina superior derecha \textrightarrow Seleccionar la opción \dq{Import...} \textrightarrow Seleccionar \dq{Clipboard} y pegar el contenido del archivo \texttt{dashboard.json} 
    \item Repetir el paso anterior, con el archivo \texttt{mock.json} 
    \item Dirigirse a la dirección \texttt{localhost:1880/ui} para visualizar el tablero de control 
\end{itemize}

El primer \emph{flow} que se carga es el correspondiente al tablero de control; define los \emph{inputs MQTT} (elementos MQTT que se suscriben a un tópico determinado, indicando la IP del servidor y el puerto), algunas funciones de transformación de valores, y los elementos de interfaz gráfica (niveles, agujas, mensajes, notificaciones). El segundo \emph{flow} genera valores aleatorios y los publica en los distintos tópicos a los cuales están suscriptos los elementos del tablero de control.  



%%%%%%%%%%%%%%%%%%%%%%%%%%%%%%%%%%%%%%%%%%%%
% FIN DOCUMENTO, AHORA REFERENCIAS
%%%%%%%%%%%%%%%%%%%%%%%%%%%%%%%%%%%%%%%%%%%%
\clearpage
\printbibliography

\end{document}
