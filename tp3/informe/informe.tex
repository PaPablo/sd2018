\input{project_settings}
\input{listings_settings}

\begin{document}

%%%%%%%%%%%%%%%%%%%%%%%%%%%%%%%%%%%%%%%%%%%%
% En 'titlepage.tex' se encuentra la página de título
%%%%%%%%%%%%%%%%%%%%%%%%%%%%%%%%%%%%%%%%%%%%
\input{titlepage}

%%%%%%%%%%%%%%%%%%%%%%%%%%%%%%%%%%%%%%%%%%%%
% INDICE
%%%%%%%%%%%%%%%%%%%%%%%%%%%%%%%%%%%%%%%%%%%%
\clearpage
\tableofcontents
\clearpage 

\lstset{style=MyStyle}

\section{Ejercicio 1 - Analizando respuesta de CGI}

Para el siguiente experimento, se utilizaron los \emph{scripts} disponibles en la carpeta \dq{ej1/}. 

En el directorio \dq{http/} se encuentra un \emph{script} para levantar un servidor HTTP utilizando la librería estándar de Python, y una petición HTTP como ejemplo. Para correr el servidor, se debe ejecutar el siguiente comando:

\begin{lstlisting}
   ./run.sh 
\end{lstlisting}

Luego, para hacer la petición, se debe utilizar \texttt{nc}, \texttt{ncat}, o \texttt{netcat} de la siguiente forma:

\begin{lstlisting}
   cat req.txt | nc localhost 8000 
\end{lstlisting}

La cabecera HTTP (que es lo que se quiere ver en este experimento) esperable es la siguiente

\begin{lstlisting}[title={Cabecera HTTP de la respuesta obtenida del servidor}]
HTTP/1.0 200 OK
Server: SimpleHTTP/0.6 Python/3.5.2
Date: Fri, 19 Oct 2018 14:47:58 GMT
Content-type: text/html; charset=utf-8
Content-Length: 373
\end{lstlisting}

Con el servidor CGI, el procedimiento para correr el servidor y realizar una petición es exactamente igual al anterior, con la siguiente respuesta esperable:

\begin{lstlisting}[title={Respuesta obtenida del servidor CGI}]
HTTP/1.0 200 Script output follows
Server: SimpleHTTP/0.6 Python/3.5.2
Date: Fri, 19 Oct 2018 14:50:03 GMT
Content-type: text/html

<h1>Hola! Probando CGI</h1>
\end{lstlisting}

Comparando la primer línea de cada respueta, se puede ver que la del servidor CGI indica que la respuesta a continuación es lo que devolvió el \emph{script} ejecutado, revelando así que es un servidor CGI.

\section{Ejercicio 2 - Aplicación de alumnos con CGI}

La aplicación desarrollada para el almacenamiento y consulta de los alumnos cuenta con las siguientes características:

\begin{itemize}
    \item Registro de un nuevo alumno
    \item Ingreso al sistema, con el número de legajo y contraseña
    \item Modificación de los datos personales
    \item Listado de los alumnos registrados
    \item Consulta por nombre, legajo (puede ser un rango o un valor específico), sexo, y edad (puede ser un rango o un valor específico)
\end{itemize}

Fue desarrollada con Python3 para los \emph{scripts} de CGI, utilizando una base de datos \emph{SQLite3} (librería estándar de Python) y JavaScript para darle un mayor comportamiento a las páginas devueltas por los \emph{scripts} de CGI.

\subsection{Requisitos}

Para correrla, como medida previa es necesario contar con los siguientes programas instalados:

\begin{itemize}
    \item \textbf{Python3} - lenguaje con el cual se desarrollaron los \emph{scripts} CGI  
    \item \textbf{pip} - gestor de paquetes de Python 
    \item \textbf{npm} - gestor de paquetes de JavaScript
\end{itemize}

\subsection{Cómo correrla}

Simplemente se debe ingresar al directorio \dq{ej2/}, instalar las dependencias necesarias con el comando \texttt{pip install -r requirements.txt} y luego correr el \emph{script} \dq{run.sh} con el comando \texttt{./run.sh}. Este script se encarga de levantar un servidor HTTP que provee la librería estándar de Python, el cual también se puede configurar como servidor CGI con el parámetro \texttt{--cgi}.

Para ingresar al sistema, en un navegador (preferentemente Firefox) dirigirse a la dirección \dq{localhost:8080}.

\subsection{\emph{Scripts} CGI}

Se implementaron los siguientes \emph{scripts} CGI:

\begin{itemize}
    \item \textbf{cgi-bin/index.py}: página principal. Muestra la tabla de alumnos y los campos para filtrar una vez que se ingresó al sistema.
    \item \textbf{cgi-bin/signup.py}: registro del alumno. Muestra un formulario para crear un nuevo alumno, haciendo controles de integridad de los datos (el largo del nombre no debe ser mayor a 70 caracteres, por ejemplo), y que el legajo no exista previamente.
    \item \textbf{cgi-bin/login.py}: página con formulario para ingreso al sistema. Genera una \emph{cookie} de sesión para no tener que reingresar los datos sucesivamente. 
    \item \textbf{cgi-bin/logout.py}: cierre de sesión. Invalida la \emph{cookie} generada al ingresar al sistema.
\end{itemize}

\subsection{Almacenamiento de alumnos}

Como se comentaba al comienzo de la presente sección, la aplicación utiliza una base de datos \emph{SQLite3}. La cual cuenta con una única tabla \dq{Alumno}, cuya definición se puede encontrar en \dq{schema/Alumno.sql}

\subsection{Generación de dinámica de HTML}

En un primer momento, la ventaja de CGI frente a los servidores HTTP era la generación dinámica de páginas web, pudiendo encrustar en un bloque de código HTML el resultado de una consulta a una base de datos.

Para la presente aplicación, en vez de imprimir el código HTMl directamente, se utilizó el \emph{motor de plantillas} de Python, \textbf{Jinja2}. Este permite definir \emph{templates}, que son estructuras HTML que reciben datos externos, con los cuales completar la página. Utiliza una sintaxis muy similar a la de Python, renderizando datos de forma condicional, iterando sobre listas, incluyendo otros \emph{templates}, y extenderlos. Las plantillas definidas para la aplicación se encuentran en el directorio \dq{templates}.

Teniendo el motor de plantillas, se pudo renderizar la tabla presente en la página inicial de la aplicación. Cabe destacar que también se utilizaron las librerías de JavaScript, \textbf{JQuery} y \textbf{DataTables}, para la visualización de la tabla.  

\subsection{¿Es esta una aplicación web?}

Esta aplicación desarrollada encaja con la definición de Aplicación Web.

Posee una arquitectura cliente-servidor muy común en este tipo de aplicaciónes. En nuestro caso, el usuario utiliza como cliente el navegador, introduciendo la URL de nuestra aplicación, que el servidor interpreta y decide qué contenido devolver. Luego, ya cuando la aplicación está funcionando, el usuario, a través de clics o envíos de formulario realiza pedidos de información o servicios al servidor. El servidor recibe los pedidos, los resuelve y le devuelve al cliente lo requerido, o un mensaje de error de no haber podido completar.


\section{Ejercicio 5 - Chat con CGI}

A partir del servidor de chat implementado en C que se presentaba en el apunte de Ajax provisto por la cátedra, se implementó el mismo servidor en Python. Luego se fueron incluyendo el resto de las funcionalidades requeridas.

A diferencia del ejercicio anterior, el chat utiliza funcionalidades de Ajax para enviar mensajes y actualizar la ventana principal, en vez de enviar un formulario y recargar la página.

\subsection{Requisitos}

Los mismos que la aplicación de la sección anterior

\subsection{Cómo correrla}

Esta aplicación no usa ninguna dependencia nueva de la anterior, por lo tanto se puede correr igual que la anterior. Etilizando el script \emph{run.sh}, se debe ejecutar con el comando \texttt{./run.sh} 

\subsection{Almacenamiento del chat y usuarios}

En este caso se obvió la base de datos y directamente se almacenó los mensajes intercambiados en el chat y los usuarios registrados en archivos de texto plano. Al igual que el servidor implementado en C, al abrir el archivo se intenta bloquearlo, para evitar condiciones de carrera y resultados inconsistentes. Para bloquear el recurso se utilizó el módulo de la librería estándar de Python \texttt{fcntl}. 

Cuando un usuario ingresa al chat, el servidor crea una nueva \textbf{sesión} para dicho usuario, devolviéndole en una \emph{cookie} el id de la sesión generado automáticamente por el servidor. En los sucesivos mensajes que intercambian el cliente y el servidor, este último se vale del id de la sesión que viaja en la \emph{cookie} del cliente para verificar su identidad. En dicha sesión el servidor también almacena cuál fue la última línea del chat que se le envío al cliente.

En la ventana principal del chat, en la parte izquierda se pueden ver los mensajes que intercambian los distintos usuarios, mientras que en la parte derecha se pueden ver los \textbf{usuarios conectados}. 

\subsection{Actualización del chat}

Tanto para el envío de un nuevo mensaje como para la actualización del chat y de los usuarios registrados, se utilizó Ajax. Se definió un intervalo de 2 segundos para ejecutar la actualización. \emph{Para ver el código JavaScript del chat, dirigirse al archivo ej5/static/js/src/chat.js}.


\section{Ejercicio 6 - DFS}

El hecho de poder compartir archivos entre dos computadores es extremadamente útil, sobre todo en ámbitos administrativos donde se dispone de una red común entre todos los equipos. Es por eso que desde Linux y desde Windows se ofrece la posibilidad de realizar de manera transparente.

\subsection{Servidor Linux - Cliente Linux - NFS}

Disponemos de un servidor corriendo con Linux y un cliente que desea tener acceso a los archivos también corriendo en Linux. Para ilustar este procedimiento se utilizó Ubuntu Server 18.04.

\subsubsection{Servidor}

Para poder realizar la compartición de archivos a través de NFS, es necesario instalar el paquete \texttt{nfs-kernel-server} utilizando apt.

\begin{lstlisting}
# update previo para actualizar la lista de paquetes
sudo apt update && \
sudo apt install nfs-kernel-server
\end{lstlisting}

Luego creamos nuestra carpeta a compartir, en nuestro caso será \texttt{/nfsshare}, con:

\begin{lstlisting}
mkdir /nfsshare
\end{lstlisting}

Luego es necesario modificar el archivo \texttt{/etc/exports} en el cual se listarán las carpetas a compartir, IP de los clientes posibles que pueden acceder y opciones de acceso. Junto con la carpeta y los posibles clientes, se pueden especificar opciones:

\begin{itemize}
    \item ro: read-only access
    \item rw: read-write access
    \item sync: Confirma las peticiones de acceso al directorio compartido solamente cuando los cambios se han confirmado.
    \item no\_subtree\_check: Evita que se verifique todo el subarbol superior al directorio accedido para verificar permisos, le agrega confiabilidad pero le quita seguridad. 
    \item no\_root\_squash: Permite al usuario \texttt{root} conectarse
\end{itemize}

    Otras opciones se detallan en la página man para \texttt{export} 

\begin{lstlisting}
# Permite a cualquier host conectarse a la carpeta compartida
/nfsshare *(rw,sync,no_subtree_check,no_root_squash)

# Permite a 192.168.2.10 conectarse en modo de solo lectura
/nfsshare 192.168.2.10(ro)

# Permite a cualquier host de la red 192.168.0.0/24 conectarse en modo de solo lectura
# Junto con el host 192.168.1.10 con acceso de lecto-escritura
/nfsshare 192.168.0.0(ro) 192.168.1.10(rw)
\end{lstlisting}

Se recomiendo reiniciar el servicio \texttt{nfs-kernel-server} una vez modificado el arhivo \texttt{/etc/exports} 

\begin{lstlisting}
sudo service nfs-kernel-server restart
\end{lstlisting}

\subsubsection{Cliente}

Para poder acceder a los archivos de la carpeta compartida es necesario, quizas no, instalar el paquete \texttt{nfs-common} 


\begin{lstlisting}
sudo apt update && \
sudo apt install -y nfs-common
\end{lstlisting}

Para poder listar las carpetas disponibles por nfs que posee un servidor podemos hacer:

\begin{lstlisting}
sudo showmount -e <ip_servidor>
\end{lstlisting}

Ya sabiendo qué carpetas se encuentran disponibles en el servidor, necesitamos una carpeta sobre la cual montar nuestro NFS, por lo tanto:

\begin{lstlisting}
sudo mkdir /mnt/nfsshare
sudo mount -t nfs <ip_servidor>:/nfsshare /mnt/nfsshare
\end{lstlisting}

Y ya tendremos nuestra carpeta disponible para uso en el cliente. Cuando no se desee usar más la carpeta haciendo \texttt{umount} de la carpeta será suficiente:

\begin{lstlisting}
sudo umount /mnt/nfsshare
\end{lstlisting}

\subsection{Windows - Linux - Samba}

Puede suceder que se disponga de un servidor de archvios corriendo en Windows y, como usuarios Linux, queremos acceder a los archivos presentes allí. Para resolver esta necesidad es que existe Samba, que es un servidor que provee acceso a Active Directory, compartición de archivos y servicios de impresión. Es extremadamente útil para comunicar equipos con Windows con equipo con Linux.

\subsubsection{Servidor Windows - Cliente Linux}

En el servidor Windows suponemos la existencia de una carpeta ya compartida con permisos de acceso al usuario Everyone (este usuario es como comodín), y que el usuario dueño de la carpeta posee contraseña de acceso. Luego en nuestro cliente Linux debemos instalar \texttt{cifs-utils} con:

\begin{lstlisting}
sudo apt update && \
sudo apt install -y cifs-utils
\end{lstlisting}

Una vez instalado corremos:

\begin{lstlisting}
mkdir $HOME/windows-share
mount.cifs //<ip-servidor-windows>/<Nombre-de-carpeta> $HOME/windows-share -o user=<usuario-dueño>
\end{lstlisting}

Y se nos pedirá la contraseña del usuario. Y ya tenemos acceso a los archivos en la carpeta compartida.

\subsubsection{Servidor Windows - Cliente Linux}

En este segundo caso los datos se encuentran en un servidor Linux y queremos acceder dese un cliente Windows. Instalamos sambda y creamos nuestra carpeta a compartir:

\begin{lstlisting}
sudo apt update && \
sudo apt install -y samba && \
mkdir $HOME/shared
\end{lstlisting}

Luego editamos el archivo \texttt{/etc/samba/smb.conf} y agregamos la siguiente entrada

\begin{lstlisting}
[<folder_name>]
path = /<path>/<to>/<folder>
available = yes

# Usuarios validos, si se deja en blanco, se permite a cualquiera
valid users = <user_name>

read only = no 
browsable = yes 
public = yes 
writable = yes
\end{lstlisting}

Llenamos con los datos que corresponden y reiniciamos el servicio de samba:

\begin{lstlisting}
sudo service restart smbd
\end{lstlisting}

En el cliente windows creamos un nuevo Acceso Directo y como destino le ponemos:

\begin{lstlisting}
\\<ip-servidor>\<folder_name>
\end{lstlisting}

Y se creará nuestro enlace a la carpeta compartida. Ahora podemos utilizarla de manera transparente en nuestro cliente Windows.

\section{Ejercicio 7 - DNS}

Realizamos visitas a las siguientes paginas, en orden, y realizamos una captura con \texttt{Wireshark} para visibilizar las consultas a DNS realizadas.

\begin{itemize}
    \item \texttt{www.sra.org.ar} - Sitio de la Sociedad Rural Argentina
    \item \texttt{https://dm.takaratomy.co.jp/} - Sitio del juego de cartas Duel Masters - Sitio de origen japonés
\end{itemize}

Luego, de la captura realizada podemos extraer los siguientes tiempos.

Al realizar la primera visita a \texttt{www.sra.org.ar} observamos que entre la petición por la IP del dominio y su respuesta hay una diferencia de 313ms.

Luego, al realizar la visita a \texttt{https://dm.takaratomy.co.jp/}, la diferencia es de 203ms. Pero el tiempo de carga de la página aumentó considerablemente en comparación con la página de la sociedad rural.

En la segunda visita a \texttt{www.sra.org.ar} el tiempo de respuesta fue similar al aterior, pero la página cargó aún más rápido que la vez anterior.

Podemos concluir que en una primera instancia la ubicación física de la referencia en un servidor DNS de la IP correspondiente a un dominio es relevante. Luego de haber realizado una consulta, ese tiempo puede bajar debido a que se almacenó en caché de algún servidor intermedio. 

Aún así, al hacer pedidos sucesivos y cercanos en el tiempo con el navegador a una misma página, los tiempos de carga disminuyen visiblemente debido a que la gran mayoría de la página se almacena en caché del propio navegador.
%%%%%%%%%%%%%%%%%%%%%%%%%%%%%%%%%%%%%%%%%%%%
% FIN DOCUMENTO, AHORA REFERENCIAS
%%%%%%%%%%%%%%%%%%%%%%%%%%%%%%%%%%%%%%%%%%%%
\clearpage
\printbibliography

\end{document}
