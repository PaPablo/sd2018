\input{project_settings}
\input{listings_settings}

\begin{document}

%%%%%%%%%%%%%%%%%%%%%%%%%%%%%%%%%%%%%%%%%%%%
% En 'titlepage.tex' se encuentra la página de título
%%%%%%%%%%%%%%%%%%%%%%%%%%%%%%%%%%%%%%%%%%%%
\input{titlepage}

%%%%%%%%%%%%%%%%%%%%%%%%%%%%%%%%%%%%%%%%%%%%
% INDICE
%%%%%%%%%%%%%%%%%%%%%%%%%%%%%%%%%%%%%%%%%%%%
\clearpage
\tableofcontents
\clearpage 

\lstset{style=MyStyle}

\section{Ejercicio 1 - Analizando respuesta de CGI}

Para el siguiente experimento, se utilizaron los \emph{scripts} disponibles en la carpeta \dq{ej1/}. 

En el directorio \dq{http/} se encuentra un \emph{script} para levantar un servidor HTTP utilizando la librería estándar de Python, y una petición HTTP como ejemplo. Para correr el servidor, se debe ejecutar el siguiente comando:

\begin{lstlisting}
   ./run.sh 
\end{lstlisting}

Luego, para hacer la petición, se debe utilizar \texttt{nc}, \texttt{ncat}, o \texttt{netcat} de la siguiente forma:

\begin{lstlisting}
   cat req.txt | nc localhost 8000 
\end{lstlisting}

La cabecera HTTP (que es lo que se quiere ver en este experimento) esperable es la siguiente

\begin{lstlisting}
HTTP/1.0 200 OK
Server: SimpleHTTP/0.6 Python/3.5.2
Date: Fri, 19 Oct 2018 14:47:58 GMT
Content-type: text/html; charset=utf-8
Content-Length: 373
\end{lstlisting}

Con el servidor CGI, el procedimiento para correr el servidor y realizar una petición es exactamente igual al anterior, con la siguiente respuesta esperable:

\begin{lstlisting}
HTTP/1.0 200 Script output follows
Server: SimpleHTTP/0.6 Python/3.5.2
Date: Fri, 19 Oct 2018 14:50:03 GMT
Content-type: text/html

<h1>Hola! Probando CGI</h1>
\end{lstlisting}



%%%%%%%%%%%%%%%%%%%%%%%%%%%%%%%%%%%%%%%%%%%%
% FIN DOCUMENTO, AHORA REFERENCIAS
%%%%%%%%%%%%%%%%%%%%%%%%%%%%%%%%%%%%%%%%%%%%
\clearpage
\printbibliography

\end{document}
