%%%%%%%%%%%%%%%%%%%%%%%%%%%%%%%%%%%%%%%%%%%%
% En 'inclues.tex' se encuentran la importación de paquetes necesarios
%%%%%%%%%%%%%%%%%%%%%%%%%%%%%%%%%%%%%%%%%%%%
\input{project_settings}
\input{listings_settings}


\begin{document}

%%%%%%%%%%%%%%%%%%%%%%%%%%%%%%%%%%%%%%%%%%%%
% En 'titlepage.tex' se encuentra la página de título
%%%%%%%%%%%%%%%%%%%%%%%%%%%%%%%%%%%%%%%%%%%%
\input{titlepage}

%%%%%%%%%%%%%%%%%%%%%%%%%%%%%%%%%%%%%%%%%%%%
% INDICE
%%%%%%%%%%%%%%%%%%%%%%%%%%%%%%%%%%%%%%%%%%%%
\clearpage
\tableofcontents
\clearpage 

\lstset{style=cstyle}

\section{Ejercicio 1}

\subsection{Inciso \emph{a}}

\emph{Los siguientes códigos fuente se pueden encontrar en el directorio ej-1/a/.} 

El código fuente del servidor fue modificado para que se mantenga vivo (siga corriendo después de desconectarse el cliente), y para que reciba el mensaje del cliente, y devuelva ese mismo mensaje, todo en mayúsculas (para ello se definió la función \texttt{strupper}). El servidor recibe por parámetro en qué puerto escuchar por conexiones entrantes.

El cliente recibe dos parámetros: el nombre del servidor (se obtiene su dirección utilizando la función \texttt{gethostbyname}), y el puerto de escucha. Una vez conectado, imprime un mensaje por pantalla pidiendo un texto para enviar al servidor, y luego imprime la respuesta recibida del servidor. El cliente manda todos los mensajes que se le ingresan, hasta cerrar la conexión con un \texttt{EOF}. 

\subsection{Inciso \emph{b}}



Al igual que en el ejercicio anterior, el servidor recibe un puerto sobre el cual escuchar, y el cliente recibe un nombre de servidor al cual intentar conectarse y un puerto. Cada uno recibe además una cantidad; este valor indica cuántos \emph{bytes} escribir en el \emph{socket} de conexión (cliente), y cuántos \emph{bytes} leer del \emph{socket} de conexión (servidor).   











%%%%%%%%%%%%%%%%%%%%%%%%%%%%%%%%%%%%%%%%%%%%
% FIN DOCUMENTO, AHORA REFERENCIAS
%%%%%%%%%%%%%%%%%%%%%%%%%%%%%%%%%%%%%%%%%%%%
\clearpage
\printbibliography

\end{document}

